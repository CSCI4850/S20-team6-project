% PACKAGES INCLUDED HERE 
% DO NOT NEED TO CHANGE
\documentclass[conference]{IEEEtran}
%\IEEEoverridecommandlockouts
% The preceding line is only needed to identify funding in the first footnote. If that is unneeded, please comment it out.
\usepackage{cite}
\usepackage{amsmath,amssymb,amsfonts}
\usepackage{algorithmic}
\usepackage{graphicx}
\usepackage{textcomp}
\def\BibTeX{{\rm B\kern-.05em{\sc i\kern-.025em b}\kern-.08em
    T\kern-.1667em\lower.7ex\hbox{E}\kern-.125emX}}
\begin{document}

% TITLE GOES HERE

\title{Paper Title*\\}


% AUTHOR NAMES GOES HERE

\author{\IEEEauthorblockN{1\textsuperscript{st} Given Name Surname}
\IEEEauthorblockA{\textit{dept. name of organization (of Aff.)} \\
\textit{name of organization (of Aff.)}\\
City, Country \\
email address}
\and
\IEEEauthorblockN{2\textsuperscript{nd} Given Name Surname}
\IEEEauthorblockA{\textit{dept. name of organization (of Aff.)} \\
\textit{name of organization (of Aff.)}\\
City, Country \\
email address}
\and
\IEEEauthorblockN{3\textsuperscript{rd} Given Name Surname}
\IEEEauthorblockA{\textit{dept. name of organization (of Aff.)} \\
\textit{name of organization (of Aff.)}\\
City, Country \\
email address}
\and
\IEEEauthorblockN{4\textsuperscript{th} Given Name Surname}
\IEEEauthorblockA{\textit{dept. name of organization (of Aff.)} \\
\textit{name of organization (of Aff.)}\\
City, Country \\
email address}
\and
\IEEEauthorblockN{5\textsuperscript{th} Given Name Surname}
\IEEEauthorblockA{\textit{dept. name of organization (of Aff.)} \\
\textit{name of organization (of Aff.)}\\
City, Country \\
email address}
\and
\IEEEauthorblockN{6\textsuperscript{th} Given Name Surname}
\IEEEauthorblockA{\textit{dept. name of organization (of Aff.)} \\
\textit{name of organization (of Aff.)}\\
City, Country \\
email address}
}

\maketitle

% ABSTRACT 

\begin{abstract}
This document is a model and instructions for \LaTeX.
This and the IEEEtran.cls file define the components of your paper [title, text, heads, etc.]. *CRITICAL: Do Not Use Symbols, Special Characters, Footnotes, 
or Math in Paper Title or Abstract.
\end{abstract}


% KEYWORDS

\begin{IEEEkeywords}
component, formatting, style, styling, insert
\end{IEEEkeywords}

% INTRODUCTION SECTION
\section{Introduction}

When you have information that could possibly harm you or others do you leave it out in the open or keep it hidden? And if you were to hide it, would you be worried that someone could find it? Well, what if there was a way to hide information that even if they found it, they could not use it? There is a way and it is how cryptography came to be. Cryptography is the art of writing or solving codes and where there is cryptography, you can find stenography. Defined as the science of hiding a secret message within an ordinary message. The reason these two go hand in hand is because where you have hidden messages, you need ways to break the code to reveal the hidden information.
This can often be very challenging as stenography is very difficult to crack because without knowing how it was encoded the information will be lost to you. This makes an importance part of cybersecurity and malware alike. Stenography is used by government agencies, individuals, and groups because of its “robustness (resistance to various image processing methods and compression) and capacity of the hidden data…” (School of Computing and Intelligent Systems).
These groups use stenography to hide their data from prying eyes, and some are being used to hide malware to steal information. This type of stenography can be regularly seen in phishing emails that contain an image or link that when clicked on releases a virus into the computer and steals information.
In order to interpret the images we are processing and decode the bits hidden inside them we will be using Cycle GANs. “Generative adversarial networks (GANs) are algorithmic architectures that use two neural networks, pitting one against the other (thus the “adversarial”) in order to generate new, synthetic instances of data that can pass for real data.” They are widely used in image generation and will help in deciphering our data. This strategy of deciphering was originally introduced in a paper by Ian Goodfellow with his fellow researches at the University of Montreal in 2014. (PathMind)
Developing this architecture requires a discriminator and a generator model. The discriminator is a Convolutional Neural Network that determines whether the received image is real or generated. While the generator uses inverse convolutional layers to create an image by transforming its input.
This network will use Cycle GANs to hide a specified image inside of another and learn to recognize the hidden image to output it. For this, we will have two sets of photos. One set will contain the images to hide, and the second set will contain images to hide the requested photo in. In order to do this the two parts of our Cycle GAN, the discriminator, and the generator, must work together to recognize real verses generated images. To accomplish this, the generator will take the most significant bits from photos in the first set and attempt to hide the image inside an image from the second set. The Discriminator will then take the generators output and photos from the second set in order to determine the likelihood that it is not a generated image. This likelihood is calculated as the loss of the network and is used to determine how accurate it is at hiding and recognizing hidden images.
This paper discusses the value and methodology of hiding images inside of another and is worth every consideration. It is worth consideration because of its possibility. If images can be accurately stored and retrieved from another it would give technology a new way to store items. You could have many images hidden inside one ready for easy transfer and "unzip" the image to reveal all the hidden ones. This process could also allow for a new way to send data as information could be converted down to bits and placed inside an image and translated when it is received. In so doing, it could simplify  \cite{b1}.

% BACKGROUND SECTION
\section{Background}

Start typing here \cite{b2}.

% METHODS SECTION
\section{Methods}

Start typing here \cite{b3}.

% RESULTS SECTION
\section{Results}

Start typing here \cite{b4}.

% DISCUSSION SECTION
\section{Discussion}

Start typing here \cite{b5}.

% REFERENCES
% THIS IS CREATED AUTOMATICALLY
\bibliographystyle{IEEEtran}
\bibliography{References} % change if another name is used for References file

\end{document}
