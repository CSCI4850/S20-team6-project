% PACKAGES INCLUDED HERE 
% DO NOT NEED TO CHANGE
\documentclass[conference]{IEEEtran}
%\IEEEoverridecommandlockouts
% The preceding line is only needed to identify funding in the first footnote. If that is unneeded, please comment it out.
\usepackage{cite}
\usepackage{amsmath,amssymb,amsfonts}
\usepackage{algorithmic}
\usepackage{graphicx}
\usepackage{textcomp}
\def\BibTeX{{\rm B\kern-.05em{\sc i\kern-.025em b}\kern-.08em
    T\kern-.1667em\lower.7ex\hbox{E}\kern-.125emX}}
\begin{document}

% TITLE GOES HERE

\title{Stenography GAN: Cracking Stenography with Cycle Generative Adversarial Networks}


% AUTHOR NAMES GOES HERE

\author{\IEEEauthorblockN{1\textsuperscript{st} Nibraas Khan}
\IEEEauthorblockA{\textit{Department of Computer Science} \\
\textit{Middle Tennessee State University}\\
Murfreesboro, TN \\
nak2z@mtmail.mtsu.edu}
\and
\IEEEauthorblockN{2\textsuperscript{nd} Ruj Haan}
\IEEEauthorblockA{\textit{Department of Computer Science} \\
\textit{Middle Tennessee State University}\\
Murfreesboro, TN \\
gm3g@mtmail.mtsu.edu}
\and
\IEEEauthorblockN{3\textsuperscript{rd} George Boktor}
\IEEEauthorblockA{\textit{Department of Computer Science} \\
\textit{Middle Tennessee State University}\\
Murfreesboro, TN \\
gsb3c@mtmail.mtsu.edu}
\and
\IEEEauthorblockN{4\textsuperscript{th} Michael McComas}
\IEEEauthorblockA{\textit{Department of Computer Science} \\
\textit{Middle Tennessee State University}\\
Murfreesboro, TN \\
mrm8m@mtmail.mtsu.edu}
\and
\IEEEauthorblockN{5\textsuperscript{th} Ramin Daneshi}
\IEEEauthorblockA{\textit{Department of Computer Science} \\
\textit{Middle Tennessee State University}\\
Murfreesboro, TN \\
rd3s@mtmail.mtsu.edu}
}

\maketitle

% ABSTRACT 

\begin{abstract}
This document is a model and instructions for \LaTeX.
This and the IEEEtran.cls file define the components of your paper [title, text, heads, etc.]. *CRITICAL: Do Not Use Symbols, Special Characters, Footnotes, 
or Math in Paper Title or Abstract.
\end{abstract}


% KEYWORDS

\begin{IEEEkeywords}
component, formatting, style, styling, insert
\end{IEEEkeywords}

% INTRODUCTION SECTION
\section{Introduction}

Talk about the importance of cyrpto. 

The sten specifically. 

Sten used in real life.

There have crack to crack cyrpto and steno. 

We are also taking on the same task. Our approach is cycle gans. 

Before that, we need to about gans in general. 

History of cycle gans 

Pix2Pix and CNN

Why will this work/or why is this worth exploring 

In order to understand its need to compare it to other models, and in this case autoencoders

For maximum accruacy we also introduced. Bayesian optimization which is ...

Start typing here \cite{b1}.

% BACKGROUND SECTION
\section{Background}

The details of the sten algo. Include some images. Give an example. 

We used cycles gans to crack this algo. That etails: 

Regular GANs. Include images and the math

Cycle GANs. Explain it a little bit 

Our implemention of cycle gans ultalizes pix2pix

Explain what pix2pix is. Include images and rough outlay of the math 

Pix2pix emplyed CNN. What they are and the math behind it. 

CipherGANs. Explain it a little bit

Autoencoders. Explain it a little bit 

Baysian optimzation. Explain it a little but

% METHODS SECTION
\section{Methods}

Math for the cycle gans

Math behind the autoencoders 

Math bethind the bayes opt

Training CycleGAN: 
    The training and testing data
    How we actually flow through the network 
    Bayes opt
    
Training Autoencoders:
    The training and testing data
    How we actually flow through the network
    
Testing protocols: 
    Testing
    
Different training techniques:
    Bit size

% RESULTS SECTION
\section{Results}

Here are the result for cycle gan: 
     Grab images from the cycle gan algo

Here are the results for autoencoder:
     Grab images from the autoencoder section 
     
Here are the results for messing with different bit size:
     Images from bit = 7, 6, 5, 4, 3

Here are the results for cycle gan with bayes opt:

% DISCUSSION SECTION
\section{Discussion}

Summary of the introduction 

If our model was successful. Why was it not successful. 

Extra things that we can work on. 

Our model presentes many fruitful avenues of research 

% REFERENCES
% THIS IS CREATED AUTOMATICALLY
\bibliographystyle{IEEEtran}
\bibliography{References} % change if another name is used for References file

\end{document}
